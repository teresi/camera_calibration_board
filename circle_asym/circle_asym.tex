% Camera calibration board, asymmetric circles

% author: Michael Teresi

\documentclass{article}

\usepackage[
	nohead,
	nofoot,
	margin=0mm,
	voffset=0mm,
	hoffset=0mm
]
{geometry}

\usepackage{tikz, ifthen}
\usetikzlibrary{fit, calc}

% \convertto{<unit>}{<dimension>}
%     convert a dimension to a new unit of measurement
% Args:
%     <unit>:      output unit of measurement to convert to (e.g. mm)
%     <dimension>: input value to convert (e.g. 1pt)
\makeatletter
%https://tex.stackexchange.com/questions/8260/what-are-the-various-units-ex-em-in-pt-bp-dd-pc-expressed-in-mm
%http://groups.google.com/group/comp.text.tex/msg/7e812e5d6e67fcc5
\def\convertto#1#2{\strip@pt\dimexpr #2*65536/\number\dimexpr 1#1}
\makeatother

% 1 pt in LaTeX is 1 / 72.27 inch
% 2.845 pt/mm = 72.27 pt / 1 inch * 1 inch / 25.4 mm
%\def\ptsPerMm{2.845275591}
\def\ptsPerMm{\convertto{pt}{1mm}}

% user input
% TODO calculate margin in order to center the circles? (distance from paper edge to circle)
\setlength{\paperwidth}{254mm}
\setlength{\paperheight}{304.8mm}

\newcommand\gridSize{64}   % circle delta from left to right (not the diagonal)
%\newcommand\sqrt2{1.414213562}
\newcommand\diam{24}       % circle diameter
\newcommand\margin{1*\diam} % minimum space to edge of page
\newcommand\sqrtTwo{1.414213562}

% TODO use convertto instead for dimensions
\pgfmathsetmacro\radius{\diam/2}
\pgfmathsetmacro\paperWmm{\paperwidth/\ptsPerMm}
\pgfmathsetmacro\paperHmm{\paperheight/\ptsPerMm}

\pgfmathsetmacro\deltaCol{\gridSize*\sqrtTwo}

\pgfmathsetmacro\deltaRow{\gridSize/\sqrtTwo}


% the case where the no. circles should decrease by one b/c of the shift
% is handled below
\pgfmathsetmacro\diag{\gridSize/\sqrtTwo}

\pgfmathtruncatemacro\numX{(\paperWmm-\margin*2)/\gridSize}
\pgfmathsetmacro\xoffset{(\paperWmm-\numX*\gridSize-\diam)/2}

\pgfmathtruncatemacro\numY{(\paperHmm-\margin*2-\diag)/\gridSize}
\pgfmathsetmacro\yoffset{(\paperHmm-\numY*\gridSize-\gridSize/2-\diam)/2}



\begin{document}
\thispagestyle{empty}

\begin{tikzpicture}
[%
	x=1mm,
	y=1mm,
	overlay,
	remember picture,
	scale=1,
	shift={(current page.north west)}
]

% NB first we draw the grid
% then draw again with then offset of 1/2 grid with one less column
% b/c the diagram *must* not be symmetric in order for openCV to find the orientation
% e.g. think of it that the 'corners' are on the top, and the 'rounded corners' are on the bottom

\pgfmathsetmacro\maxColDistDraw{0}
\foreach \colIndex [evaluate={\colDist=\colIndex*\gridSize+\xoffset+\radius}] in {0,...,\numX}
\foreach \rowIndex [evaluate={\rowDist=\rowIndex*\gridSize+\yoffset+\radius}] in {0,...,\numY}
{
	\coordinate (A) at (\colDist, -\rowDist);
	\fill [black] (A) circle (\radius);
}

\pgfmathsetmacro\numXminusone{\numX-1}
\foreach \colIndex [evaluate={\colDist=\colIndex*\gridSize+\xoffset+\radius}] in {0,...,\numXminusone}
\foreach \rowIndex [evaluate={\rowDist=\rowIndex*\gridSize+\yoffset+\radius}] in {0,...,\numY}
{
	\coordinate (A) at (\colDist+\gridSize/2, -\rowDist-\gridSize/2);
	\fill [black] (A) circle (\radius);
}

\pgfmathsetmacro\lineWidth{\gridSize}
\pgfmathsetmacro\textXPose{\gridSize*2+\margin}
\pgfmathsetmacro\textYPose{\gridSize*-0.1}
\pgfmathsetmacro\textMargin{\gridSize*0.1}
\pgfmathtruncatemacro\numYInt{int(\numX)+1}
\pgfmathtruncatemacro\numXInt{int(\numY)+1}

% NB by OpenCV convention, the row count is the no. of rows w/o the 1/2 grid offset
% but the no. of cols includes the 2nd grid drawn at the 1/2 offset
% MAGIC adding 1 to each of these b/c it's a count not an index
\pgfmathtruncatemacro\numRows{\numY+1}
\pgfmathtruncatemacro\numCols{\numX+1+\numXminusone+1}
%

\node[font=\footnotesize] at (\xoffset+\diam, -\yoffset-\numY*\gridSize-\gridSize)
{
	$ \numRows\; x\; \numCols \; $
	Board,
	\pgfmathprintnumber[fixed, precision=1]{\paperWmm}
	$x$
	\pgfmathprintnumber[fixed, precision=1]{\paperHmm}
	$mm$
};



%
%\draw[|<->|] (\margin+\deltaCol/2, \textYPose) -- node[]{} ++(\deltaCol, 0);
%\node[align=left, font=\footnotesize] at (\margin+\deltaRow/2, \textYPose)
%{
%	$\Delta = \pgfmathprintnumber[fixed, precision=2]{\deltaCol} mm$
%};
%

%\draw[|<->|] (\xoffset+\radius+\gridSize*1/4, -\yoffset-\radius-\gridSize*3/4) -- node[]{} ++(\gridSize/2, -\gridSize/2);
%\node[font=\footnotesize, rotate=-45] at (\margin*1.2+\deltaRow*3/2, -\margin*1.2-\deltaRow*3/2)
%{
%	$\Delta = \diag$
%};

% top left corner, diameter
\draw[|<->|] (\xoffset, -\margin) -- node[]{} ++(\diam, 0);
\node[align=left, font=\footnotesize] at (\xoffset+1.6*\diam, -\margin)
{
	$\phi = \diam mm$
};


 right of diameter, spacing
\draw[|<->|] (\xoffset+\gridSize+\radius, -\margin) -- node[]{} ++(\gridSize, 0);
\node[font=\footnotesize] at (\xoffset+\gridSize*2+\diam, -\margin)
{
	$\Delta = \gridSize mm$
};


\end{tikzpicture}

\end{document}
\grid
