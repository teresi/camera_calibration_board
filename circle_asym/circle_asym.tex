% Camera calibration board, asymmetric circles

% author: Michael Teresi

\documentclass{article}

\usepackage[
	nohead,
	nofoot,
	margin=0mm,
	voffset=0mm,
	hoffset=0mm
]
{geometry}

\usepackage{tikz, ifthen}
\usetikzlibrary{fit, calc}

% \convertto{<unit>}{<dimension>}
%     convert a dimension to a new unit of measurement
% Args:
%     <unit>:      output unit of measurement to convert to (e.g. mm)
%     <dimension>: input value to convert (e.g. 1pt)
\makeatletter
%https://tex.stackexchange.com/questions/8260/what-are-the-various-units-ex-em-in-pt-bp-dd-pc-expressed-in-mm
%http://groups.google.com/group/comp.text.tex/msg/7e812e5d6e67fcc5
\def\convertto#1#2{\strip@pt\dimexpr #2*65536/\number\dimexpr 1#1}
\makeatother

% 1 pt in LaTeX is 1 / 72.27 inch
% 2.845 pt/mm = 72.27 pt / 1 inch * 1 inch / 25.4 mm
%\def\ptsPerMm{2.845275591}
\def\ptsPerMm{\convertto{pt}{1mm}}

% user input
% TODO calculate margin in order to center the circles? (distance from paper edge to circle)
\setlength{\paperwidth}{300mm}
\setlength{\paperheight}{300mm}

\newcommand\gridSize{50}   % circle delta
%\newcommand\sqrt2{1.414213562}
\newcommand\diam{30}       % circle diameter
\newcommand\margin{\diam} % minimum space to edge of page
\newcommand\sqrtTwo{1.414213562} % minimum space to edge of page

% TODO use convertto instead for dimensions
\pgfmathsetmacro\radius{\diam/2}
\pgfmathsetmacro\paperWmm{\paperwidth/\ptsPerMm}
\pgfmathsetmacro\paperHmm{\paperheight/\ptsPerMm}
\pgfmathsetmacro\deltaCol{\gridSize*\sqrtTwo}
\pgfmathsetmacro\deltaRow{\gridSize/\sqrtTwo}

\pgfmathsetmacro\numY{(\paperHmm-\margin*2)/\deltaRow}
% the case where the no. circles should decrease by one b/c of the shift
% is handled below
\pgfmathsetmacro\numX{(\paperWmm-\margin*2)/\deltaCol}


\begin{document}
\thispagestyle{empty}

\begin{tikzpicture}
[%
	x=1mm,
	y=1mm,
	overlay,
	remember picture,
	scale=1,
	shift={(current page.north west)}
]

\pgfmathsetmacro\maxColDistDraw{0}
\foreach \colIndex [evaluate={\colDist=\colIndex*\deltaCol}] in {0,...,\numX}
\foreach \rowIndex [evaluate={\rowDist=\rowIndex*\deltaRow}] in {0,...,\numY}
{
	\ifodd\rowIndex
		\coordinate (A) at (\colDist+\margin, -\rowDist-\margin);
		\pgfmathsetmacro\maxColDistDraw{\colDist+\diam/2}
		\fill[black] (A) circle (\diam/2);
	\else
		\coordinate (A) at (\colDist+\margin+\deltaRow, -\rowDist-\margin);
		\pgfmathsetmacro\maxColDistDraw{\colDist+\deltaRow+\diam/2}
	\fi

	\pgfmathparse{\maxColDistDraw > (\paperWmm-\margin) ? 1 : 0}
	% if the circle won't be drawn over the edge of the page, draw the circle
	\ifthenelse{\pgfmathresult = 0}{\fill [black] (A) circle (\diam/2);}{;}
}

\pgfmathsetmacro\lineWidth{\gridSize}
\pgfmathsetmacro\textXPose{\gridSize*2+\margin}
\pgfmathsetmacro\textYPose{\gridSize*-0.1}
\pgfmathsetmacro\textMargin{\gridSize*0.1}
\pgfmathtruncatemacro\numYInt{int(\numX)+1}
\pgfmathtruncatemacro\numXInt{int(\numY)+1}

\node[font=\footnotesize] at (\paperwidth/2+\deltaCol*2, \textYPose)
{
	$ \numXInt \; x\; \numYInt \; $
	Board
	\pgfmathprintnumber[fixed, precision=1]{\paperWmm}
	$x$
	\pgfmathprintnumber[fixed, precision=1]{\paperHmm}
	$mm$
};

\draw[|<->|] (\margin+\deltaCol/2, \textYPose) -- node[]{} ++(\deltaCol, 0);
\node[align=left, font=\footnotesize] at (\margin+\deltaRow/2, \textYPose)
{
	$\Delta = \pgfmathprintnumber[fixed, precision=2]{\deltaCol} mm$
};

\draw[|<->|] (\margin+\deltaRow/2, -\margin-\deltaRow/2) -- node[]{} ++(\deltaRow, -\deltaRow);
\node[font=\footnotesize, rotate=-45] at (\margin*1.2+\deltaRow*3/2, -\margin*1.2-\deltaRow*3/2)
{
	$\Delta = \lineWidth mm$
};

\draw[|<->|] (\textMargin, -\margin+\radius) -- node[]{} ++(0, -\diam);
\node[align=left, font=\footnotesize, rotate=90] at (\textMargin, -\margin-\diam)
{
	$\phi = \diam mm$
};


\end{tikzpicture}

\end{document}
\grid
