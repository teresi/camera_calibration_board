% Camera calibration board with circles

% author: Michael Teresi

\documentclass{article}

\usepackage[
	nohead,
	nofoot,
	margin=0mm,
	voffset=0mm,
	hoffset=0mm
]
{geometry}

\usepackage{tikz, ifthen}
\usetikzlibrary{fit, calc}

% \convertto{<unit>}{<dimension>}
%     convert a dimension to a new unit of measurement
% Args:
%     <unit>:      output unit of measurement to convert to (e.g. mm)
%     <dimension>: input value to convert (e.g. 1pt)
\makeatletter
%https://tex.stackexchange.com/questions/8260/what-are-the-various-units-ex-em-in-pt-bp-dd-pc-expressed-in-mm
%http://groups.google.com/group/comp.text.tex/msg/7e812e5d6e67fcc5
\def\convertto#1#2{\strip@pt\dimexpr #2*65536/\number\dimexpr 1#1}
\makeatother

% 1 pt in LaTeX is 1 / 72.27 inch
% 2.845 pt/mm = 72.27 pt / 1 inch * 1 inch / 25.4 mm
%\def\ptsPerMm{2.845275591}
\def\ptsPerMm{\convertto{pt}{1mm}}

% user input
% TODO calculate margin in order to center the circles? (distance from paper edge to circle)
\setlength{\paperwidth}{300mm}
\setlength{\paperheight}{300mm}

\newcommand\squareSize{54}   % circle delta
\newcommand\diam{40}       % circle diameter
\newcommand\margin{\diam*1} % minimum space to edge of page

\pgfmathsetmacro\paperWmm{\paperwidth/\ptsPerMm}
\pgfmathsetmacro\paperHmm{\paperheight/\ptsPerMm}
\pgfmathsetmacro\numX{(\paperWmm-\margin*2)/\squareSize}
\pgfmathsetmacro\numY{(\paperHmm-\margin*2)/\squareSize}

\begin{document}
\thispagestyle{empty}

\begin{tikzpicture}
[%
	x=1mm,
	y=1mm,
	overlay,
	remember picture,
	scale=1,
	shift={(current page.north west)}
]

\foreach \colIndex [evaluate={\colDist=\colIndex*\squareSize}] in {0,...,\numX}
\foreach \rowIndex [evaluate={\rowDist=\rowIndex*\squareSize}] in {0,...,\numY}
{
	\coordinate (A) at ( \colDist + \margin, -\rowDist - \margin);
	\fill [black] (A) circle (\diam/2);
}

\pgfmathsetmacro\lineWidth{\squareSize}
\pgfmathsetmacro\textXPose{\squareSize*2+\margin}
\pgfmathsetmacro\textYPose{\squareSize*-0.1}
\pgfmathsetmacro\textMargin{\squareSize*0.1}
\pgfmathtruncatemacro\numYInt{int(\numX)+1}
\pgfmathtruncatemacro\numXInt{int(\numY)+1}

% print the board dimensions and number of circles
\node[align=left, font=\footnotesize] at (\textXPose, \textYPose)
{
	$\numXInt \; x\; \numYInt \;$
	Board
	\pgfmathprintnumber[fixed, precision=1]{\paperWmm}
	$x$
	\pgfmathprintnumber[fixed, precision=1]{\paperHmm}
	$mm$
};

% print 
\draw[|<->|] (\margin, \textYPose) -- node[below, font=\footnotesize]
{ $\Delta = \lineWidth mm$} ++(\lineWidth, 0);

\draw[|<->|] (-\textYPose, -\margin+\diam/2) -- node[]{} ++(0, -\diam);
\node[left, font=\footnotesize, rotate=90] at (-\textYPose, -\margin-\diam/1.75)
{ $\phi = \diam mm$  };

\end{tikzpicture}

\end{document}
\grid
