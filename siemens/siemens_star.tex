% siemens star (aka spoke target) focus pattern

% author: Michael Teresi


% NB setting \paperwidth after geometry produced an error
%    where tikz page center was not actually in center of page
%    pass options on command line instead and use `current page.center`


\documentclass[oneside]{article}

% NB do not set paperwidth, paperheight here b/c they are NOT
%    overridden by the command line options
\usepackage[
	ignoreall,
	hmargin=0pt,
	vmargin=0pt,
	noheadfoot,
	nomarginpar,
	offset=0pt,
]
{geometry}

\usepackage{tikz, ifthen}
\usetikzlibrary{fit, calc}

% \convertto{<unit>}{<dimension>}
%     convert a dimension to a new unit of measurement
% Args:
%     <unit>:      output unit of measurement to convert to (e.g. mm)
%     <dimension>: input value to convert (e.g. 1pt)
\makeatletter
%https://tex.stackexchange.com/questions/8260/what-are-the-various-units-ex-em-in-pt-bp-dd-pc-expressed-in-mm
%http://groups.google.com/group/comp.text.tex/msg/7e812e5d6e67fcc5
\def\convertto#1#2{\strip@pt\dimexpr#2*65536/\number\dimexpr 1#1}
\makeatother

% 1 pt in LaTeX is 1 / 72.27 inch
% 2.845 pt/mm = 72.27 pt / 1 inch * 1 inch / 25.4 mm
%\def\ptsPerMm{2.845275591}
\def\ptsPerMm{\convertto{pt}{1mm}}

\setlength{\paperwidth}{86.36mm}
\setlength{\paperheight}{53.34mm}

\begin{document}

% provides input variables
% nSpokes    number of filled sections in the star, aka spoke target, (45)
% radius     radius in mm of the spokes (45% of pagewidth)
\IfFileExists{options}{% replace like so
% sed -i 's/.*%radius/88%radius/g' options.tex


% MAGIC 45 spokes (8 deg/cycle) is common
\newcommand\nSpokes{
45%nSpokes
}


%\newcommand\radius{40}
}{}

\ifdefined\nSpokes
\else
	\newcommand\nSpokes{45}  % MAGIC 8deg / spoke appears to be common
\fi

\ifdefined\radius
\else
	% MAGIC user prefence, 90 of page with
	\pgfmathsetmacro\radius{\paperwidth*0.48/\ptsPerMm}
\fi


% draw squares eminating from center
% usefule for finding center of page / positioning
%\begin{tikzpicture}[remember picture,overlay]
%	\begin{scope}[color=gray,line width=1pt]
%\foreach \shift in {1cm,2cm,3cm,4cm,5cm,6cm,7cm,8cm,9cm,10cm,11cm,12cm,13cm,14cm}
%	  \draw ([xshift=-\shift,yshift=\shift]current page.center) rectangle ([xshift=\shift,yshift=-\shift]current page.center);
%	\end{scope}
%\end{tikzpicture}


\pgfmathsetmacro\angle{360.0/2/\nSpokes}  % spokes fill *half* the circle
\begin{tikzpicture}
[%
	x=1mm,
	y=1mm,
	overlay,
	remember picture,
	scale=1,
]
\foreach \spokeIndex [evaluate={\spoke=(\spokeIndex*2*\angle)}] in {0,...,\nSpokes}
{
	\pgfmathsetmacro\start{\spoke-\angle/2}
	\pgfmathsetmacro\stop{\spoke+\angle/2}

	\filldraw [fill=black, ultra thin]
		(current page.center)
		++({\radius*cos(\start)}, {\radius*sin(\start)}) arc(\start:\stop:\radius)
		-- (current page.center)
		-- cycle;
};
% MAGIC experimental circle radius, depends on your printer
\filldraw[white] (current page.center) circle (0.1pt);
\end{tikzpicture}%


\end{document}
\grid
